
\section{Scene files}
% For commands from original Colobot there is old documentation in the resources/ directory. We need to copy and verify that. ~krzys_h

Scene files are an important type of files that describe all playable worlds. Each scene file contains full specification of one world. Understanding the structure of these files is crucial for those who wish to create new planets and levels.

Scene files are relatively easy to understand text files that specify how world is constructed and what it contains. This chapter describes a current version of scene files, not fully compatible with scene files used by original Colobot game. Incompatibilities will be presented later in text.


\subsection{Syntax and basic structure}

\todo[inline]{Add info about default values for all commands}

Scene files are basically a list of commands that control how the world is generated. Each command is put in separate line. Empty lines are ignored. Text that starts with \texttt{//} characters is treated as a comment and also ignored. Comments are useful when you want to describe some sections in scene file or temporarily disable something.

\begin{verbatim}
// this is comment, below is disabled command
//CreateObject pos=10.0;20.0 dir=0.5 type=NuclearCell
\end{verbatim}

Each command starts with a name that is written in upper camel case, e.g. \texttt{AmbientColor}. A command might contain a list of key-value pairs that specify parameters. A key is a string that specify a name of parameter and value contains a specific string that will be used to initialize parameter. Keys are separated from values by \texttt{=} character and each pair is separated by any number of white space characters. For example, command \texttt{AmbientColor} can have a parameters \texttt{air} and \texttt{water} with values \texttt{0.471;0.353;0.000;0.000} and \texttt{0.078;0.078;0.078;0.078} respectively. A full command will look like below:

\begin{verbatim}
AmbientColor air=0.471;0.353;0.000;0.000 water=0.078;0.078;0.078;0.078
\end{verbatim}

A scene file can be divided into few easily recognizable parts:

\begin{itemize}
    \item Initialization
    \item World specification
    \item Object creation
    \item Finalization
\end{itemize}

In strings, you can use the following special values:

% TODO: why do these % signs look so weird? Bad font?
\begin{description}
    \item[\texttt{\%lvl\%}] resolves to level directory
    \item[\texttt{\%chap\%}] resolves to chapter directory
    \item[\texttt{\%cat\%}] resolves to category directory \todo{I don't remember if this is implemented or not}
    \item[\texttt{\%lng\%}] resolves to current language char (see the fragment about translating commands below)
\end{description}

Commands can also be defined as language-specific. That is, command will be executed only for language specified in that command. Language is specified using a dot and language-specific character in command's name, for example \texttt{Title.E} is \texttt{Title} command that will only be executed for English language. Currently, these are the languages you can use:

\begin{description}
    \item[E] English
    \item[F] French
    \item[P] Polish
    \item[D] German
    \item[R] Russian
\end{description}

If language is not defined, it defaults to English. Lack of language specification means that command will be executed for all languages. You can put multiple language specifications in one scene file, for example you can put multiple \texttt{Title} commands to give scene titles for multiple languages, like so:

\begin{verbatim}
Title.E text="Example title"
Title.P text="Przykładowy tytuł"
\end{verbatim}


\subsection{Initialization}

The first part is initialization. In this phase world is given a title, description, and is associated with text files that describe instructions for player, satellite information, solution, and CBot manual.

Title and description are specified on per language basis. Each scene file can be created with various languages in mind. The first two commands in file are usually defined like below:

\begin{verbatim}
Title.# text="Title for this mission"
Resume.# text="Short description of this mission"
\end{verbatim}

\texttt{Title} defines a title that will be shown in level menu and \texttt{Resume} defines a description that will be shown when map is highlighted in level menu.

Next part is where you specify files to be used for instructions, solution and CBot manual, how game ending looks like, and how long messages appear on screen.

Instructions with mission objectives are specified using \texttt{Instructions} command. Satellite information about weather condition, aliens, types and quantities of ores below and above ground are specified with \texttt{Satellite} command. \texttt{Loading} is used for programs sent by Houston mission control to help you complete mission objectives. Solution can be defined using \texttt{SoluceFile} command and \texttt{HelpFile} defines CBot manual file. Each one takes only one parameter (\texttt{name}) that specify path to file that contains mission objectives. Commands will most often be used like below:

% Broken syntax highlighting :/
\begin{verbatim}
Instructions name="%lvl%/help/help.%lng%.txt"
Satellite name="%lvl%/help/report.%lng%.txt"
Loading name="%lvl%/help/prog.%lng%.txt"
SoluceFile name="%lvl%/help/soluce.%lng%.txt"
HelpFile name="cbot.txt"
\end{verbatim}

Next command is used to define ending screen for winning and failing mission. It has two integer parameters and looks like below:

\begin{verbatim}
EndingFile win=# lost=#
\end{verbatim}

Each number is actually used to create file name to another scene file in ''other'' directory that will be used when mission is completed or failed. For example if \texttt{win} is 1, then scene ''win001.txt'' will be used. Value -1 has special meaning -- mission is simply ended and game goes back to mission menu screen.

\todo[inline]{This behavior might be changed in the future. For now it does not allow custom ending files to be used unless you put your files to ''other'' directory.}

Finally, there's an optional command for defining audio track for background music. Track number is used to create file name, for example track 3 means music file ''music003.ogg''.
\begin{verbatim}
Audio track=3
\end{verbatim}

\todo[inline]{Scene version 2 has additional parameters, verify and add description here} % didn't we decide that we won't support version 1 anymore and that we'll remove versioning system?

\todo[inline]{AudioChange}

Optional command \texttt{MessageDelay} is used to change how long messages are displayed on screen. Default factor 1 means 5 seconds. If you want to change the time messages are displayed to 10 seconds, you can use following definition:

\begin{verbatim}
MessageDelay factor=2
\end{verbatim}


\subsection{World specification}
\todo[inline]{Complete}

Next phase is world specification. Here we define how world is being built, the relief, underground, resources, various colors, etc. Colors are defined as a sequence of floating point values separated by semicolon. Together they specify RGBA color. Value 0 means 0 in this particular part, 1 means full value.
% in original colobot it was 0-255

\todo[inline]{Incompatibilities with original scene files}


\subsubsection{World colors}

First color command is \verb|AmbientColor|. It is used to define the global color for world. It has two parameters, \verb|air| defines above water ambient color and \verb|water| defines underwater ambient color.

\begin{verbatim}
AmbientColor air=0.471;0.353;0.000;0.000 water=0.078;0.078;0.078;0.078
\end{verbatim}

\todo[inline]{In original scene files command was named AmbiantColor}

Command \verb|FogColor| defines color of fog. It also has two parameters like previous command.

\begin{verbatim}
FogColor air=0.392;0.392;0.353;0.000 water=0.263;0.314;0.392;0.000
\end{verbatim}

\verb|VehicleColor| is used to define color for various objects, including vehicles, buildings and some items. Defaults to gray. This command has only one parameter \verb|color|.

\begin{verbatim}
VehicleColor color=0.659;0.620;0.463;0.000
\end{verbatim}

Next we can define color of plants in our world.

\begin{verbatim}
GreeneryColor color=0.631;0.592;0.161
\end{verbatim}

We can also change default color of aliens.

\begin{verbatim}
InsectColor color=0.000;0.721;0.100
\end{verbatim}


\subsubsection{View and sky commands}

We define how far we can see. Command \texttt{DeepView} has two parameters, above water and underwater visibility in meters. Command \texttt{FogStart} also has two parameters, each one defines how far fog starts. Lower value means denser fog.

\begin{verbatim}
DeepView air=50.00 water=25.00
FogStart air=0.1 water=0.5
\end{verbatim}

\texttt{SecondTexture} command defines which dirt texture is used. It's only parameter defines index to dirt textures, 0 means no dirt texture, 1-8 means specific dirt texture.

\begin{verbatim}
SecondTexture rank=1
\end{verbatim}
\todo[inline]{That has now an alternative syntax with path to image file}

After that we define default background used for sky. This command has multiple definitions. One version defines simply two colors that will be blended to create simple sky effect.

\begin{verbatim}
Background up=0.392;0.392;0.353;0.000 down=0.392;0.392;0.353;
\end{verbatim}

We can also define a gradient for clouds using two additional parameters, \texttt{cloudUp} and \texttt{cloudDown}.

\todo[inline]{Show example of correct usage}

More complex sky uses additional \texttt{image} parameter. It will be used as long as sky is enabled in options. If not, sky will be defined by color gradient.

\begin{verbatim}
Background image="back01.png" up=0.392;0.392;0.353;0.000 down=0.392;0.392;0.353;
\end{verbatim}

Lens flare effect is defined using \texttt{FrontsizeName} command. You only specify image file to be used for this effect.
\todo[inline]{Wasn't this renamed in Gold?}

\begin{verbatim}
FrontsizeName image="lens5.png"
\end{verbatim}

Moving planets and other space objects can be defined using \texttt{Planet} command. It takes a number of parameters:

\begin{description}
    \item[\texttt{mode}] \todo{Complete}
    \item[\texttt{pos}] Defines starting position in Colobot radius
    \item[\texttt{dim}] Defines size of the object
    \item[\texttt{speed}] Defines how fast object is moving, 0 means static object
    \item[\texttt{dir}] Defines direction of movement
    \item[\texttt{image}] Image file used as texture
    \item[\texttt{uv1}] First UV
    \item[\texttt{uv2}] Second UV
\end{description}

2 UVs are used because one file can contain multiple planets or other objects. Example of definition is shown below:
\todo[inline]{Better description of UVs with an image?}
\todo[inline]{How does alpha channel work?}

\begin{verbatim}
Planet mode=0 pos=1.2; 0.2 dim=0.40 speed= 0.000 dir=0.5 image="planet02.png"
    uv1=0.0;0.0 uv2=0.5;0.5
\end{verbatim}


\subsubsection{Terrain commands}

\texttt{TerrainGenerate} defines how far terrain is visible, it's depth and hardness. \texttt{vision} has to be at least twice the value defined in \texttt{DeepView}.
%TODO: Does it? I don't really remember

\begin{verbatim}
TerrainGenerate vision=400.00 depth=1 hard=0.5
\end{verbatim}

Wind velocity is defined in command below. \texttt{speed} parameter has two float values. First one defines speed from north to south. Second one defines speed from west to east. Negative value defines reversed direction.

\begin{verbatim}
TerrainWind speed=0.0;2.0
\end{verbatim}

\todo[inline]{Changing wind changes direction of flags (I think), note about changing clouds}

You can give your planet electromagnetic discharges effect like the one on Orpheon. This is defined using \texttt{TerrainBlitz} command. \texttt{sleep} is used to define delay. Delay is a time before first lightning occurs, which gives player a chance to build proper protection. \texttt{delay} defines time in seconds between lightnings. \texttt{magnetic} defines damaging radius. If robot or building is withing this radius from lightning, it will be destroyed. Value of 0 effectively disables all damage, which means that lightnings are harmless.\todo{Add info about what buildings can be destroyed (not all!) and a probability for this building to be chosen (this is different for some!)}

\begin{verbatim}
TerrainBlitz sleep=60 delay=3 magnetic=50
\end{verbatim}

Terrain shape itself is defined using command \texttt{TerrainRelief}. It takes two parameters, one being file containing grayscale image defining a heightmap and second being height factor. Factor of 1.0 means that each change of color is 0.25 meters in terrain.
\todo[inline]{Make sure this is correct, I think some params in TerrainCreate also can change that}

\begin{verbatim}
TerrainRelief image="relief26.png" factor=1.0
\end{verbatim}

\texttt{TerrainResource} command specifies resource map that defines where in the world you can find ores, energy, and underground keys.

\begin{verbatim}
TerrainResource image="res12.png"
\end{verbatim}

You can define water parameters in world. You do that with \texttt{TerrainWater} command. It has a number of parameters:

\begin{description}
    \item[\texttt{image}] Defines image for water texture
    \item[\texttt{level}] Defines water level in meters
    \item[\texttt{moveX}] Horizontal movement speed 1\todo{on X axis}
    \item[\texttt{moveY}] Horizontal movement speed 2\todo{on Y axis}
    \item[\texttt{color}] Water color
    \item[\texttt{brightness}] Water brightness\todo{What does this change?}
\end{description}

\begin{verbatim}
TerrainWater image="water16.png" level=30.0 moveX=1.0 moveY=0.0
color=0.000;0.941;0.392;0.000 brightness=0.2
\end{verbatim}

Clouds can be created at certain height using command \texttt{TerrainCloud}. You specify image for texture and height level. Clouds will move according to wind.

\begin{verbatim}
TerrainCloud image="cloud04.png" level=125.0
\end{verbatim}


\subsubsection{Terrain texturing}

Now begins the texturing phase. There are two ways of defining how terrain is being textured. Simple way uses one texture with few tiles that will be applied automatically. It is useful for simple planets and moons, where terrain texture is simple and does not have special effects.

\begin{verbatim}
TerrainInitTextures image="mars" dx=4 dy=2 table=1;2;3;4;5;6;7;8
\end{verbatim}

Second method uses manual texturing and allows you to apply various tiling textures based on height and slopes. This advanced texturing allows you to define a terrain where grass is applied between height 10 and 20, and snow appears above height 50. This is indeed used in quite a few missions, such as on Terranova.

First, you need specially constructed textures which contain subtextures with transitions from one type of terrain to another. Example of such texture is shown on figure \ref{fig:advanced_texturing}.

\begin{figure}[ht]
\begin{center}

\begin{picture}(10, 10)
\put(0, 0){ Need picture here }
\end{picture}

\caption{Example of a texture that can be used for advanced terrain texturing}
\label{fig:advanced_texturing}

\end{center}
\end{figure}

You need to add an ID number to each texture file; this ID will be referenced later. Then you use multiples of the following command:

\begin{verbatim}
TerrainMaterial id=1 image="%user%\texture.png" u=0.00 v=0.00 up=1 down=1
left=1 right=1 hard=0.5
\end{verbatim}

\texttt{TerrainMaterial} command takes a number of arguments:

\begin{description}
    \item[id] this is texture ID; it can be omitted if previous command defined it
    \item[image] this is texture file name
    \item[u, v] these define UV within texture that define where subtexture starts; subtexture's size is always 25\% of images size
    \item[up, down, left, right] these define the neighboring tiles, e.g. if you put \texttt{up=1}, it means that this tile has texture with ID 1 above
    \item[hard] hardness of terrain material, defines the sound of footsteps
\end{description}

In order to fully define terrain textures, you need to have 15 or 16 \texttt{TerrainMaterial} commands per texture. Below is a valid example of terrain texturing using one texture file.

\begin{verbatim}
TerrainMaterial id=1 image="%user%\texture.png" u=0.00 v=0.00 up=1 down=1
left=1 right=1 hard=0.5
TerrainMaterial      image="%user%\texture.png" u=0.25 v=0.00 up=2 down=1
left=1 right=1 hard=0.3
TerrainMaterial      image="%user%\texture.png" u=0.50 v=0.00 up=1 down=1
left=1 right=2 hard=0.3
TerrainMaterial      image="%user%\texture.png" u=0.75 v=0.00 up=2 down=1
left=1 right=2 hard=0.3
TerrainMaterial      image="%user%\texture.png" u=0.00 v=0.25 up=1 down=2
left=1 right=1 hard=0.3
TerrainMaterial      image="%user%\texture.png" u=0.25 v=0.25 up=2 down=2
left=1 right=1 hard=0.3
TerrainMaterial      image="%user%\texture.png" u=0.50 v=0.25 up=1 down=2
left=1 right=2 hard=0.3
TerrainMaterial      image="%user%\texture.png" u=0.75 v=0.25 up=2 down=2
left=1 right=2 hard=0.3
TerrainMaterial      image="%user%\texture.png" u=0.00 v=0.50 up=1 down=1
left=2 right=1 hard=0.3
TerrainMaterial      image="%user%\texture.png" u=0.25 v=0.50 up=2 down=1
left=2 right=1 hard=0.3
TerrainMaterial      image="%user%\texture.png" u=0.50 v=0.50 up=1 down=1
left=2 right=2 hard=0.3
TerrainMaterial      image="%user%\texture.png" u=0.75 v=0.50 up=2 down=1
left=2 right=2 hard=0.3
TerrainMaterial      image="%user%\texture.png" u=0.00 v=0.75 up=1 down=2
left=2 right=1 hard=0.3
TerrainMaterial      image="%user%\texture.png" u=0.25 v=0.75 up=2 down=2
left=2 right=1 hard=0.3
TerrainMaterial      image="%user%\texture.png" u=0.50 v=0.75 up=1 down=2
left=2 right=2 hard=0.3
TerrainMaterial id=2 image="%user%\texture.png" u=0.75 v=0.75 up=2 down=2
left=2 right=2 hard=0.2
\end{verbatim}

Once you define your textures, you need to apply them to terrain. First command you need for that is \texttt{TerrainInit}. It is used to define default texture. It uses texture ID defined in \texttt{TerrainMaterial}.

\begin{verbatim}
TerrainInit id=1
\end{verbatim}

Then we can apply terrain texture variations using \texttt{TerrainLevel} command. It takes a number of parameters:

\begin{description}
    \item[id] texture ID from \texttt{TerrainMaterial}
    \item[min] (optional) minimum height in meters for this texture to appear
    \item[max] (optional) maximum height in meters for this texture to appear
    \item[slope] (optional) defines terrain slope angle from -10 to 10; positive values are for more smooth and plain terrain, negative values are for steep terrain
    \item[center, radius] (optional) defines circular place in terrain where this texture will be applied; means entire world if omitted
    \item[freq] frequency of this texture in percent (needs \% sign at the end)
\end{description}

For example, you can define a grass texture with ID 2 that will cover 60\% of the terrain between height 10 and 40 but on smooth terrain using command below:

\begin{verbatim}
TerrainLevel id=2 min=10 max=40 slope=0.3 freq=60%
\end{verbatim}

Terrain creation is finished using command below:
\todo[inline]{I think it takes additional parameters, I need to check if it was added in Gold or was it present in original}

\begin{verbatim}
TerrainCreate
\end{verbatim}


\subsubsection{Ground spots}

You can perform coloring of terrain in certain places using \texttt{GroundSpot} command. There are two forms. First one places a ground spot in specific position and radius.

\begin{verbatim}
GroundSpot pos= 0; 0 radius=50 color=0.392;0.784;0.902
\end{verbatim}

Second form colors entire terrain at specific heights. Heights are defined as range from \texttt{min} to \texttt{max} in meters. \texttt{smooth} defines transition from normal colors to spot color.

\begin{verbatim}
GroundSpot color=0.000;1.000;0.000 min=-10 max=31.25 smooth=10
\end{verbatim}


\subsubsection{Flying height}

You can change maximum flying height using command below:

\begin{verbatim}
MaxFlyingHeight max=70
\end{verbatim}


\subsection{Object creation}
\todo[inline]{Complete}

To start creating objects you need to issue starting command \texttt{BeginObject}. Then you use any number of commands \texttt{CreateObject} to create individual objects of various types.

The most basic form of \texttt{CreateObject} command is shown below:

\begin{verbatim}
CreateObject pos=x;y dir=d type=t
\end{verbatim}

Each object is given it's original position $(x, y)$, direction $d$ and type $t$. For example, if you want to create an astronaut at position (10,20) facing north, you want to write:

\begin{verbatim}
BeginObject
CreateObject pos=10.0;20.0 dir=0.5 type=Me
\end{verbatim}

Each object can have various other parameters that are either specific to that object type or situation.
\todo[inline]{Image describing directions}


\subsubsection{Hidden objects}

Objects can be hidden from radar\todo{and map} and marked as ''useful'' using two additional parameters. ''Useful'' object will show a message and particle effects when your unit is close enough to it. For example, if you want to make a power cell and hide it until player comes 5 meters to it, you can use following command:

\begin{verbatim}
CreateObject pos=-3.75; 3.75 dir=0.0 type=PowerCell
proxyActivate=1 proxyDistance=5
\end{verbatim}

\todo[inline]{Maybe a note about that you can actually find those using retobject() and retobjectbyid()? (this needs to be fixed by the way)}


\subsubsection{Energy cells}

Energy cells are given power level using \texttt{power} parameter. Use value between 0.0 and 1.0 to define power level. 0.0 means empty cell, 1.0 means full cell.

You can also use \texttt{power} parameter in robots to give them energy cell. There are few rules:

\begin{description}
    \item[-1] means no energy cell of any kind
    \item[0-1] means power cell with energy in given range
    \item[1-100] means nuclear cell with energy divided by 100
\end{description}


\subsubsection{Programs}

You can give programs to robots and aliens, and run them automatically. To do this you use \texttt{stript\emph{n}} parameters, where \emph{n} is program position counting from 1 to 10. To run program automatically you can use \texttt{run} parameter that specify which program should be run. Also, some programs allow you to input special arguments extracted using \emph{cmdline()} CBot function. For this you can use \texttt{cmdline} parameter. For example, to create an ant with automatic program you can use following command:

\begin{verbatim}
CreateObject pos= -55; 243 cmdline= -55; 243; -62; 234 dir=0.0
type=AlienAnt script1="ant04.txt" run=1
\end{verbatim}


\subsubsection{Destructibility}

You can change how much damage object will take using \texttt{magnifyDamage} parameter. It takes a floating point number that changes damage factor. Value below 1 means object can take more damage, value above 1 means it is more destructible. Value 0 makes object completely indestructible; not even water or lightning can destroy it. If you want to make an astronaut who is 10 times more powerful, you can use command like below:
\todo[inline]{Note that this is just a multiplier applied to all object's damage}

\begin{verbatim}
CreateObject pos=0;3 dir=0.5 type=Me magnifyDamage=0.1
\end{verbatim}


\subsubsection{Selectable and trainer objects}

\texttt{selectable} allows you to disable ability to select given object. Put a value of 0 to it. Not even \texttt{selectinsect} cheat can help you.
\todo[inline]{We need a new cheat that CAN help you}

\texttt{trainer} set to 1 changes robot to trainer type. It cannot be controlled manually, only via programming.

\todo[inline]{'manual' for Scribbler (didn't I make it just use trainer?), maybe a note about 'toy' (to be removed)}
\todo[inline]{SpaceShip with selectable=0 doesn't make you need to launch it to finish mission (for EndMissionTake section)}


\subsubsection{Space ship}

You can control space ship behavior with \texttt{run} parameter. You can use following values:

\begin{description}
    \item[0] Doors are open, nothing happens
    \item[1] Landing and opening doors
    \item[2] Movement above ground, landing and opening doors (to be used with crane\todo{Portico?})
    \item[3] Special mode for winning/losing scene files\todo{What does it do?}
    \item[11] Interplanetary movement and landing\todo{Detailed desctiption, how to set start/end planets}
\end{description}


\subsubsection{Astronaut options}

You can select specific variant of astronaut and Houston technicians using \texttt{option} parameter. Valid values:

\begin{description}
    \item[0] Default combination, has both helmet and backpack
    \item[1] Has no helmet, will die if underwater
    \item[2] Has no helmet and no backpack, can die in water
    \item[3] No body, only hands, legs, glasses and neutron gun\todo{What? Is it used anywhere?}
\end{description}


\subsubsection{Alien eggs}

Alien eggs allow you to create aliens with some delay. They have few parameters to control their behavior:
\todo[inline]{We should rename these}

\begin{description}
    \item[\texttt{autoValue1}] Delay time in seconds
    \item[\texttt{autoType}] Alien category
    \item[\texttt{autoString}] Program file name
    \item[\texttt{autoRun}] If 1, program will be automatically run
\end{description}

Example usage:

\begin{verbatim}
CreateObject pos= 98; -49 type=AlienEgg autoValue1=120 autoType=AlienSpider autoString="spider01.txt" run=1
\end{verbatim}


\subsubsection{Pyrotechnic effects}

You can give object special effects using \texttt{pyro}. Value WIN will give it firework effect, while LOST will give it smoke effect.
\todo[inline]{Does it work for all objects?}

\begin{verbatim}
CreateObject pos=10;-30 dir=0 type=Firework pyro=WIN
\end{verbatim}


\subsection{Finalization}

After all objects have been specified, there are still few things we need to define before scene file is complete. Next part is defining map parameters. This is done using two commands: \texttt{MapColor}, which defines colors used for rendering minimap, and \texttt{MapZoom} which defines default zoom and optionally allows you to disable minimap completely. Colors are specified as 3 RGB values (no alpha). \texttt{floor} means color of terrain and \texttt{water} means color of water. \texttt{factor} is zoom value which can be anything from 1 to 8. If you want to disable minimap, you can use optional parameter \texttt{enable} in \texttt{MapZoom} and set it to 0.

\begin{verbatim}
MapColor floor=0.647;0.557;0.420 water=0.604;0.922;1.000
MapZoom factor=2
\end{verbatim}

Next we can define default programs for robots. This is done using \texttt{NewScript} command. It takes two parameters, one being file name of the program and second being type of robots. Special value \texttt{All} means that program will be given to all robots.

\begin{verbatim}
NewScript name="%user%/myprogram.txt" type=WheeledGrabber
\end{verbatim}

After that we define what kind of buildings player can create using \texttt{EnableBuild} command. It takes only one parameter that specify building type.

\begin{verbatim}
EnableBuild type=ResearchCenter
\end{verbatim}

Valid building types:

\todo[inline]{Complete description of all building types?}

\begin{description}
    \item[\texttt{Flag}] Player can put and take flags
    \item[\texttt{FlatGround}] Player can check if terrain is flat
    \item[Building category] Player can build this type of buildings
\end{description}

Then we define what kind of researches are available and already done. This is done using \texttt{EnableResearch} and \texttt{DoneResearch} respectively.

\begin{verbatim}
// tracker bots are researched, winger bots are available
EnableResearch type=TRACKER
EnableResearch type=WINGER

DoneResearch type=TRACKER
\end{verbatim}

\begin{multicols}{2}[Valid research types:]

\begin{description}
    \item[\texttt{TRACKER}] Tracker bots
    \item[\texttt{WINGER}] Flying bots
    \item[\texttt{SHOOTER}] Shooters
    \item[\texttt{TOWER}] Defense tower
    \item[\texttt{ATOMIC}] Nuclear plant
    \item[\texttt{THUMPER}] Thumper bot
    \item[\texttt{SHIELDER}] Shielder bot
    \item[\texttt{PHAZER}] Phazer shooter
    \item[\texttt{iPAW}] Legged bots\textsuperscript{1}
    \item[\texttt{iGUN}] Organic shooters\textsuperscript{1}
    \item[\texttt{SUBBER}] Subber\textsuperscript{2}
    \item[\texttt{SNIFFER}] Sniffer bots\textsuperscript{2}
    \item[\texttt{RECYCLER}] Recycler bot\textsuperscript{2}
\end{description}

\end{multicols}

\textsuperscript{1} Research done in laboratory. \\
\textsuperscript{2} Cannot be researched, must be explicitly enabled.

At the whole end of scene file we define mission winning and losing conditions. This is all done using \texttt{EndMissionTake} command. It takes various parameters depending on what type of condition we define.

\begin{description}
    \item[pos] Defines center from which condition is checked
    \item[dist] Defines range under which condition is checked
    \item[type] Defines a type of objects to be checked
    \item[min] Defines minimum number of objects to win a mission
    \item[max] Defines maximum number of objects to win a mission
    \item[lost] Defines a number of objects that will cause mission failure
    \item[immediat] If set to 1, this condition is the only necessary to win a mission, other conditions are not checked
\end{description}

In all missions, you can find condition that will fail mission if an astronaut is killed. This can be defines as follows:

\begin{verbatim}
EndMissionTake pos=0;0 dist=25000.00 type=Me lost=0
\end{verbatim}

If you want to make sure player collects enough titanium ore and puts them on the spaceship, you can use following command (\texttt{dist} can be omitted):

\begin{verbatim}
EndMissionTake pos=80;-40 type=TitaniumOre min=4
\end{verbatim}

You can use optional command to delay winning and losing. Special value \texttt{-1} means immediate effect.

\begin{verbatim}
EndMissionDelay win=1 lost=2
\end{verbatim}
