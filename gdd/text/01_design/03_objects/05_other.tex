
\section{Other objects}


\subsection{Flags}

% no description in original Colobot GDD
\todo[inline]{no description in original Colobot GDD}
% "A button (+) allows you to mark a location with flags of various colors. 5 flags of 5 different colors are at your disposal. You can remove a flag by using the next button (-) and store it for further use." looks good
BlueFlag, RedFlag, GreenFlag, YellowFlag, VioletFlag


\subsection{Spots}

% no description in original Colobot GDD
\todo[inline]{no description in original Colobot GDD}
PowerSpot, TitaniumSpot, UraniumSpot


\subsection{WayPoint}

% no description in original Colobot GDD
\todo[inline]{no description in original Colobot GDD}


\subsection{Start Area and Goal Area}

The starting pad is a platform used in some programming exercises.

\begin{description}
    \item[Object category:] \texttt{StartArea}
\end{description}

The finishing pad is an objective to be reached in some programming exercises.

\begin{description}
    \item[Object category:] \texttt{GoalArea}
\end{description}


\subsection{Mine}

Mines should be left alone.
Mines were laid by the first expedition to defend against aliens. Because they caused so
many accidents, and following the pressure from the Congress and public opinion, your
expedition was not equipped with this technology.
They also appear in some programming exercises as a hard way to learn how to avoid obstacles. Mines
cannot be created or removed. They just happen to be there.
When working with a practice bot, don't get any closer than 2 meters.

\begin{description}
    \item[Object category:] \texttt{Mine}
\end{description}


\subsection{Barrier}

% no description in original Colobot GDD
\todo[inline]{no description in original Colobot GDD}


\subsection{Wreck}

% no description in original Colobot GDD
\todo[inline]{no description in original Colobot GDD}


\todo[inline]{List all decorative objects}
\subsection{Ruin}

% no description in original Colobot GDD
\todo[inline]{no description in original Colobot GDD}
